% Generic LaTeX document
% LaTeX generi preamble.tex

%\documentclass[11pt, a4paper]{article}   % a4 = 21cm x 29.7cm
\documentclass[11pt, letterpaper]{article}   % 8.5" x 11

% PACKAGES
\usepackage[top=3.5cm, bottom=3.5cm, left=3.5cm, right=3.5cm]{geometry}
%           Width of text area = 21 - 7 = 14cm = 5.512 inches
\usepackage{titlesec}
\titleformat{\chapter}[hang]
	{\normalfont\Large\bfseries}{\chaptertitlename\ \thechapter.}{10pt}{\Large}
\titleformat{\section}
	{\normalfont\large\bfseries}{\thesection}{1em}{}
\titlespacing*{\chapter} {0pt}{0in}{0.1in}
\titlespacing*{\section} {0pt}{0.15in plus 0.1in minus 0.05in}{0in}
%left-before-after {3.5ex plus 1ex minus .2ex}{2.3ex plus .2ex}
\usepackage{amsmath}
\usepackage{booktabs}
\usepackage{enumitem}
\usepackage{graphicx}
\usepackage[labelfont=bf,labelsep=newline,justification=centering]{caption}
\usepackage{parskip}
\usepackage{hyperref}\hypersetup{colorlinks = true,
                                 linkcolor = black,
								 urlcolor = black,
								 citecolor = black,
								 anchorcolor = black}

% REDEFINITIONS
% Float to top of page - tex.stackexchange.com/questions/66252/
% placing-the-figure-exactly-at-the-top-of-the-page-in-latex
\makeatletter
\setlength{\@fptop}{0pt}
\makeatother

% Rename Abstract to Executive Summary
\renewcommand{\abstractname}{Executive Summary}

% Override Latex defaults that mess up positioning of tables and figures
% mintaka.sdsu.edu/gf/bibliog/latex/floats.html
\setcounter{topnumber}{2}
\setcounter{bottomnumber}{2}
\setcounter{totalnumber}{4}
\renewcommand{\topfraction}{0.95}
\renewcommand{\bottomfraction}{0.95}
\renewcommand\floatpagefraction{.90}
\renewcommand\textfraction{.10}
\setlength{\textfloatsep}{3ex plus0.5ex minus0.2ex}

% WATERMARK - Comment following out to remove, also lines in main.tex
% Draft for review - Do not cite or quote
% tex.stackexchange.com/questions/118939/
% add-watermark-that-overlays-the-images
% tex.stackexchange.com/questions/132582/transparent-foreground-watermark
%\usepackage[printwatermark]{xwatermark}
%\usepackage{xcolor}
%\usepackage{tikz}
%\newsavebox\mybox
%\savebox\mybox{\tikz[color=red,opacity=0.5]\node{Draft for review - Do not cite or quote};}
%\newwatermark*[
%  allpages,
%  angle=0,
%  scale=2,
%  xpos=0,
%  ypos=137 
%]{\usebox\mybox}
% opacity - higher is darker - by experiment - range is presumably 0-1
% angle - 0 is horizontal
% scale is size of text - by experiment
% xpos 0 - 0 is centered
% ypos 137 - Larger number is higher on the page - by experiment


\begin{document}

Preparing Metadata for Processing IPUMS-International Samples with R

Griffith Feeney

24 December 2014


\section{Introduction}

 - Why R? Because I use it and it is the best tool for me to do this work

 - Likely to be increasingly widely used in the future by other IPUMS-I users
 
 - May be useful for future IPUMS-I operations
 
 
 \section{Vectorization}

 Will need to give some background on this, rationale, think later how to handle it. Should get a DSTIL issue out of this.
 
 
\section{Options}

Can I use stata datasets and get into R, do it this wayt? I haven't, but should check this out. Now. Extract request 48 resubmitted. Waiting. Got it. Well, wow, first thing I find is that the stata 'do' file is (a) a text file and (b) much better structured than the 'Basic codebook'. Should probably write R program to parse *it* instead.

\section{Structure of stata.do file}

Record layout information consists of lines like

\begin{verbatim}
  byte    zm10a_rooms        57-58    ///
\end{verbatim}

preceded by a line that looks like this

\begin{verbatim}
quietly infix                         ///
\end{verbatim}

and followed by a line that looks like this

\begin{verbatim}
  using `"ipumsi_00048.dat"'
\end{verbatim}

The part in quotes will change, probably not the ``using''.

Variable names and descriptions are given in lines that look like this.

\begin{verbatim}
label var zm10a_tv          `"Any television in household"'
\end{verbatim}

preceded and followed by an empty line.

Codebook information is contained in lines like this

\begin{verbatim}
label define cntry_lbl 214 `"Dominican Republic"', add
\end{verbatim}

in empty-line-separated sequences giving values for each variable with a codebook.

Well, I can begin working with this in R and document as I go. It will be MUCH easier than what I was doing before, may be fully automatable.

WOW. Okay, vastly easier to do it this way, my days of coding R to get the basic codebook information were wasted. Need a while to grok this.ol






\section{Notes}

What's the trick for getting all unharmonized variables? Got it from Bob, but have forgotten it.





\end{document}